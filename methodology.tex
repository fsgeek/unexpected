
\begin{figure}[!th]
  \captionsetup{justification=centering}
  \centering
  \caption{Polybench: DRAM vs Interleaved PMEM (both 4K pages).  Results closer to expectations but still better than expected.  Caching masks lower bandwidth of PMEM; memory usage optimization improves cache benefits which improves performance and decreases benefit of PMEM interleaving. Sorted by PMEM performance.}
  \vspace{2mm}
  \label{fig:polybench:dram-vs-ipmem-4k}
  \begin{tabular}{c}
    \includegraphics[width=0.45\textwidth]{pb-4k-DRAM-4K-iPMEM.eps}
  \end{tabular}
\end{figure}


%%
PMEM's AppDirect mode permits two types of direct-access (DAX) usage: \textit{devdax} access, which provides raw PMEM that is memory mapped by an application for exclusive use. Devdax mode requires pre-selecting memory allocation units (4KB, 2MB, and 1GB), static partitioning of the overall memory, using the ``ndctl'' utility, and is restricted to privileged applications; and \textit{fsdax} access, which uses a DAX-aware file system to provide support for sharing between multiple applications, dynamic memory allocation --- including page alignment and allocation unit size, and multi-user security.  When an application memory maps a file on a DAX-aware file system the operating system configures the process page tables to directly reference the physical addresses of the underlying PMEM.  Preferred practice is to use a DAX-aware file system~\cite{rudoff2019NVMProgrammingModel}. We used DAX-aware file systems for our investigation.
%%


%%
The Persistent Memory Development Kit (PMDK) includes libraries for transparently converting standard memory allocation calls (e.g., \textit{malloc}) to a memory allocator using arenas backed directly by files on a DAX filesystem device.  This is similar to the approach used by \textit{hugetlbfs}, a Linux file system that exposes DRAM to applications for allocation against large pages --- the standard malloc calls in that case are implemented using arenas backed directly by files on hugetlbfs.  We use both of these mechanisms in our evaluation of the Polybench tests.
%%

%%
Our PMEM hardware uses two dedicated memory controllers per CPU; each memory controller has three channels; each channel can manage two PMEM modules, which are equivalent to DRAM DIMM packages.   The memory controllers include a small cache memory and provide the ability to transparently stripe across the PMEM modules although non-interleaved access is also supported.  Each PMEM module also contains memory for converting 64 byte cache line sized load/store operations into 256 byte PMEM block size load/store operations.
%%

%%
The Linux operating system includes native support for PMEM.  However, the policies for PMEM differ from those for DRAM. The Linux kernel transparently uses large memory page mappings for PMEM whenever possible, as dictated by the alignment and length of the page mappings, falling back to smaller sizes as necessary.  Linux version 5.3 uses 1GB, 2MB, or 4KB page sizes for PMEM.  Earlier Linux versions only used 2MB or 4KB pages for PMEM. The default for DRAM remains 4KB.  We note that large page performance impact is well-described in the literature~\cite{agarwal2017thermostat,panwar2018making}. This difference accounted for PMEM performing up to 5x better than DRAM.
%%


%%
We evaluated three different DAX-aware file systems: ext and xfs, which are standard Linux file systems incorporating experimental DAX support, and NOVA. The NOVA file system was the first purpose-built file system for PMEM~\cite{xu2016nova}.  However, we found NOVA unsuitable for use as a DAX file system, because its storage allocation policy does not preserve the 2MB page alignment Linux requires for large page support; this is not reported in even recent literature~\cite{izraelevitz2019basic,yang2019empirical}. We did not expect our DAX-aware file system choice to be relevant, as once memory mapped, applications directly access the PMEM.  We used ext4 for our evaluation, as it allowed us to control PMEM alignment.
%%

\begin{figure}[!b]
  \captionsetup{justification=centering}
  \centering
  \caption{Polybench: Cases where PMEM (non-interleaved PMEM) faster than iPMEM (interleaved PMEM) --- interleaving does \textit{not} provide substantial benefit in these cases; surprising because we expect the 2MB page size to dominate.}
  \vspace{2mm}
  \label{fig:polybench:PMEM-vs-ipmem}
  \begin{tabular}{c}
    \includegraphics[width=0.45\textwidth]{pb-2M-PMEM-2MB-iPMEM-non-interleaved-faster.eps}
  \end{tabular}
\end{figure}



%%
The Polybench test suite is a set of 30 computational kernels drawn from varied domains. It is commonly used for evaluating the impact of memory performance with respect to those tasks. We previously used Polybench~\cite{doudali2017comerge}. Polybench continues to be actively used for evaluating memory optimizer performance.
%%



% Need to define this somewhere
\definecolor{Gray}{gray}{0.9}
\begin{table*}[!ht]
	\caption{Polybench Test Results: data size, by memory type (DRAM, PMEM, iPMEM), page size (4K, 2M), and optimization (\texttt{-O3} versus \texttt{-poly}).  Execution time in seconds; sorted by 4K DRAM time improvement (optimized versus poly). Only shows test where polyhedral optimization benefitted at least one memory configuration (before rounding); other results omitted. Memory locality impact differs across memory types and configurations. CPU and memory allocation bound to one NUMA node. Ratio > 1 (pre-rounding) shows polyhedral optimization benefits.}
	\label{table:poly2}
  \begin{adjustbox}{max width=\textwidth}
    \begin{tabular}{r|S|SSS|SSS|SSS|SSS|SSS|SSS}
      \multicolumn{1}{c|}{Test} &
      \multicolumn{1}{c|}{Data} &
      \multicolumn{6}{c|}{DRAM} &
      \multicolumn{6}{c|}{PMEM} &
      \multicolumn{6}{c}{iPMEM}
      \tabularnewline
      & 
      \multicolumn{1}{c|}{Size} & 
      \multicolumn{3}{c}{4K} &
      \multicolumn{3}{c|}{2M} &
      \multicolumn{3}{c}{4K} &
      \multicolumn{3}{c|}{2M} &
      \multicolumn{3}{c}{4K} &
      \multicolumn{3}{c}{2M}
      \tabularnewline
        & {MB} 
        & {Opt}  & {Poly} & {Ratio} & {Opt} & {Poly} & {Ratio} 
        & {Opt}  & {Poly} & {Ratio} & {Opt} & {Poly} & {Ratio} 
        & {Opt}  & {Poly} & {Ratio} & {Opt} & {Poly} & {Ratio}
      \tabularnewline
      \hline
      gemm & 91.6 
      & 29.3 & \textbf{1.5} & 19.1 & 14.1 & \textbf{1.5} &  9.3 
      & 39.0 & \textbf{1.7} & 22.6 & 32.0 & \textbf{1.7} & 18.7 
      & 37.8 & \textbf{1.7} & 22.9 & 27.5 & \textbf{1.7} & 16.6
      \tabularnewline
      \rowcolor{Gray}
      3mm & 120.2 
      & 17.1 & \textbf{1.5} & 11.8 & 12.1 & \textbf{1.4} & 8.4
      & 16.6 & \textbf{1.8} &  9.5 & 11.9 & \textbf{1.7} & 6.8
      & 17.1 & \textbf{1.7} & 10.3 & 11.7 & \textbf{1.6} & 7.2
      \tabularnewline
      2mm & 85.8 
      & 5.4 & \textbf{0.5} & 11.3 &  3.8 & \textbf{0.5} & 7.7
      & 5.2 & \textbf{0.6} &  8.6 &  4.1 & \textbf{0.6} & 6.9
      & 5.4 & \textbf{0.6} &  9.7 &  3.9 & \textbf{0.6} & 6.9
      \tabularnewline
      \rowcolor{Gray}
      correlation & 137.4 
      & 105.1 & \textbf{22.4} & 4.7 & 23.2 & \textbf{20.5} & 1.1 
      & 123.8 & \textbf{23.9} & 5.2 & 45.5 & \textbf{23.3} & 2.0 
      & 118.8 & \textbf{23.6} & 5.0 & 39.9 & \textbf{22.9} & 1.7
      \tabularnewline
      covariance & 49.5 
      & 19.7 & \textbf{4.7} & 4.2 & 3.3 & 4.4 & 0.8
      & 20.3 & \textbf{5.0} & 4.1 & 4.4 & 4.9 & 0.9
      & 20.0 & \textbf{4.9} & 4.1 & 4.1 & 4.8 & 0.9
      \tabularnewline
      \rowcolor{Gray}
      mvt & 1717.1 
      & 3.4 & \textbf{1.0} & 3.5 & 1.1 & \textbf{1.0} & 1.0
      & 4.7 & \textbf{2.0} & 2.3 & 2.3 & \textbf{2.0} & 1.1
      & 4.5 & \textbf{1.9} & 2.4 & 2.1 & \textbf{1.9} & 1.1
      \tabularnewline
      gemver & 1717.4 
      & 3.7 & \textbf{1.5} & 2.4 & 1.2 & 1.7 & 0.8
      & 7.9 & \textbf{4.9} & 1.6 & 5.3 & \textbf{4.9} & 1.1
      & 5.2 & \textbf{3.2} & 1.6 & 2.7 & 3.2 & 0.9
      \tabularnewline
      \rowcolor{Gray}
      jacobi-1d & 152.6 
      & 27.6 & \textbf{12.0} & 2.3 & 26.0 & \textbf{11.0} & 2.4
      & 261.8 & \textbf{70.0} & 3.7 & 260.0 & \textbf{70.1} & 3.7
      & 89.0 & \textbf{70.0} & 1.3 & 88.6 & \textbf{69.9} & 1.3
      \tabularnewline
      doitgen & 512.5 
      & 13.0 & \textbf{6.2} & 2.1 & 8.6 & \textbf{6.1} & 1.4
      & 13.3 & \textbf{10.2} & 1.3 & 12.4 & \textbf{9.8} & 1.3
      & 13.5 & \textbf{7.2} & 1.4 & 12.5 & \textbf{7.1} & 1.8
      \tabularnewline
      \rowcolor{Gray}
      gramschmidt & 91.6 
      & 67.0 & \textbf{38.8} & 1.7 & 11.3 & \textbf{9.75} & 1.2
      & 73.1 & \textbf{50.6} & 1.4 & 24.5 & 28.7 & 0.9
      & 70.5 & \textbf{44.8} & 1.6 & 16.6 & 19.1 & 0.9
      \tabularnewline
      fdtd-2d & 1464.8 
      & 33.0 & \textbf{19.3}  & 1.7 & 34.8  & \textbf{17.0} & 2.0
      & 275.5 & \textbf{57.4} & 4.8 & 273.6 & \textbf{56.9} & 4.8 
      & 104.29 & \textbf{52.2} & 2.0 & 103.8 & \textbf{52.5} & 2.0
      \tabularnewline
      \rowcolor{Gray}
      syrk & 61.0 
      & 9.1 & \textbf{5.4}  & 1.6 & 9.0 & \textbf{5.4}  & 2.0
      & 15.1 & \textbf{6.0} & 2.5 & 15.0 & \textbf{5.9} & 2.5
      & 14.6 & \textbf{6.1} & 2.4 & 15.1 & \textbf{6.1} & 2.5
      \tabularnewline
      jacobi-2d & 976.6 
      & 18.7  & \textbf{12.1} & 1.5 & 18.1  & \textbf{9.0}  & 2.0 
      & 179.6 & \textbf{62.3} & 2.8 & 179.6 & \textbf{62.1} & 2.9
      & 59.6  & 60.1          & 1.0 & 61.0  & \textbf{60.0} & 1.0
      \tabularnewline
      \rowcolor{Gray}
      syr2k & 91.6 
      & 17.9 & \textbf{12.0} & 1.5 & 18.0 & \textbf{11.8} & 1.5
      & 31.0 & \textbf{14.6} & 2.1 & 30.0 & \textbf{14.6} & 2.1
      & 26.0 & \textbf{14.3} & 2.8 & 25.9 & \textbf{14.2} & 2.8
      \tabularnewline
      symm & 91.6 
      & 49.6 & \textbf{34.2} & 1.4 & 9.8 & 11.5 & 0.9
      & 65.1 & \textbf{44.5} & 1.5 & 41.4 & \textbf{38.9} & 1.1
      & 64.4 & \textbf{44.0} & 1.5 & 37.8 & \textbf{23.5} & 1.6
      \tabularnewline
      \rowcolor{Gray}
      dynprog & 245.4 
      & 11.3 & \textbf{11.3} & 1.0 & 11.5 & 11.5          & 1.0
      & 71.4 & 73.0          & 1.0 & 74.7 & 75.3          & 1.0
      & 31.4 & 31.5          & 1.0 & 32.0 & \textbf{31.8} & 1.0
      \tabularnewline
      durbin & 1526.2 
      & 1.7 & \textbf{1.7}  & 1.0 & 0.5 & \textbf{0.5} & 1.0
      & 4.0 & 4.0           & 1.0 & 2.8 & \textbf{2.8} & 1.0
      & 2.6 & \textbf{2.6}  & 1.0 & 1.3 & 1.3          & 1.0
      \tabularnewline
      \rowcolor{Gray}
      seidel-2d & 190.7 
      & 25.1 & \textbf{25.1} & 1.0 & 24.5 & \textbf{24.5} & 1.0
      & 31.3 & \textbf{31.2} & 1.0 & 31.0 & 31.1          & 1.0
      & 25.1 & \textbf{25.1} & 1.0 & 25.1 & \textbf{25.1} & 1.0
      \tabularnewline
      trisolv & 1716.8 
      & 0.2 & 0.2          & 1.0 & 0.2 & \textbf{0.2} & 1.0
      & 0.3 & \textbf{0.3} & 1.0 & 0.3 & \textbf{0.3} & 1.0 
      & 0.3 & 0.3          & 1.0 & 0.3 & \textbf{0.3} & 1.0
      \tabularnewline
      \rowcolor{Gray}
      cholesky & 122.1 
      & 12.0 & 13.0 & 0.9 & 11.9 & \textbf{10.8} & 1.1
      & 18.8 & 20.5 & 0.9 & 18.5 & 19.4          & 1.0
      & 17.7 & 20.4 & 0.9 & 17.2 & 18.5          & 0.9
      \tabularnewline
      lu & 122.1 
      & 17.0 & 25.2            & 0.7 & 16.0 & 28.4            & 0.6
      & 209.8 & \textbf{192.2} & 1.1 & 219.2 & \textbf{192.8} & 1.1
      & 55.1 & 78.2            & 0.7 & 54.4 & 80.5            & 0.7
      \tabularnewline
      \rowcolor{Gray}
      reg\_detect & 381.6 
      & 30.1 & 52.2 & 0.6 & 28.7 & 51.0 & 0.6
      & 331.2 & \textbf{119.0} & 2.8 & 327.7 & \textbf{118.9} & 2.8
      & 93.3 & 118.6 & 0.8 & 92.6 & 118.4 & 0.8
      \tabularnewline
    \end{tabular}
  \end{adjustbox}
\end{table*}
