\ada{I prefer it when the evaluation is presented in present tense
  (except for when mentioning prior results). I do like the past tense
in the abstract of this paper, so maybe it will be weird to switch to
present tense. }

\begin{figure*}[!th]
  \captionsetup{justification=centering}
  \centering
  \caption{Polyhedral Memory optimizations for interleaved PMEM, 4K and 2M pages.  Varying cache layers impact performance differently. Results sorted by execution time without polyhedral optimization (\texttt{-polly} option to clang); boldface for improved poly performance.}
  \vspace{0.1cm}
  \label{fig:polybench:all:time}
  \begin{tabular}{cc}
    \includegraphics[width=0.45\textwidth]{pb-2M-iPMEM-poly-2M-iPMEM.eps} &
    \includegraphics[width=0.45\textwidth]{pb-4K-iPMEM-poly-4K-iPMEM.eps}
  \end{tabular}
\end{figure*}


%\input{appendix/polybench/aep/figures/tex/covariance-figure.tex}
%\input{appendix/polybench/aep/figures/tex/symm-figure.tex}
%\input{appendix/polybench/aep/figures/tex/lu-figure.tex}
%\input{appendix/polybench/aep/figures/tex/gemm-figure.tex}

%%
% We used the following tools for our evaluation: \textit{Polybench} 3.2~\cite{polybench,yuki2014understanding,yuki2015polybench}, a suite of benchmarks commonly used to evaluate memory and processor intensive workloads across a range of domains, and LLVM 10.0 --- specifically the clang compiler and polly polyhedral optimizations.

%%
Our evaluation system is a dual socket NUMA architecture system, using two Intel\textsuperscript{\textregistered} second generation Xeon\textsuperscript{\textregistered} Scalable processors (codename \textit{Cascade Lake}) running between 1.0GHz and 3.9GHz (variable) with 1.3MB L1 caches, 40MB L2 caches, and a 55MB L3 cache, 20 cores per processor, and 12 memory slots per processor, with six 32GB DRAM modules and six 256GB PMEM modules, all running at 2666MT/s. Each processor has two persistent memory controllers, with three channels per controller.  Each channel managed two PMEM modules; Intel refers to this as the 2-2-2 configuration~\cite{intel2019quickstart}. We found similar results on the same base system with different model Intel CPUs: 1-3.7GHz (variable), 32KB L1, 1MB L2, 32MB L3 caches.
%%

%%
We used Fedora 31 with Linux kernel 5.0.0, which includes native PMEM support. We configured the PMEM as six-way interleaved memory for one processor (iPMEM), and six single non-interleaved memories for the other processor (PMEM).  We used the SNIA standard programming model, which is a DAX-aware file system providing dynamic PMEM management and security; while it is possible to use devdax mode, which provides raw PMEM access, it requires static allocation and privileged (``root'') access.  We also used the PMDK libraries for transparently converting standard malloc calls into corresponding mmap calls for direct access PMEM, which only works with a DAX-aware file system~\cite{PMDK}.
%%


%%
We used Polybench/C Version 3.2~\cite{polybench,yuki2014understanding,yuki2015polybench} compiled with clang version 10.0.0 (as well as the \texttt{-mllvm -polly} memory layout optimizations), options to report execution time, and to use the Linux real-time scheduler. We matched dataset sizes with our prior work~\cite{doudali2017comerge}.  We ran the single-threaded Polybench 3.2 tests serially, bound to a single core, and all memory, including PMEM, was allocated from memory local to the NUMA node of that core.  We used the PMDK memory allocator, which redirects standard MALLOC calls to use PMEM that memory maps PMEM, via the DAX aware file system, and managed that memory with a jemalloc-based memory allocator.
%%

%%
Figure \ref{fig:polybench} reproduces our initial results.  We used default system behavior, with the NOVA file system.  The results surprised us, as they showed better performance than we expected.  We determined that while the default behavior for Linux is to use 4KB TLB mappings for DRAM, it used 2MB TLB mappings for PMEM when possible. Indeed, this behavior of Linux has changed so as of Linux 5.3 it now uses the largest possible TLB mapping for PMEM between 1GB, 2MB, and 4KB, based upon the alignment and length of the memory region being accessed.  This behavior has not previously been reported in the literature, despite its substantial impact on performance.  We counted the number of TLB and last level cache misses when forcing different page sizes; it accounts for the performance difference. Recent work has alluded to the 2MB page impact, but fails to explain why this occurs~\cite{kadekodi2019splitfs}, while other recent work fails to address it, despite the profound impact it has on performance for workloads with larger locality~\cite{dong2019performance,peng2019system}. We note that NOVA, which was designed for PMEM and is the \textit{de facto} standard for PMEM-optimized file systems interacts poorly with the page-size needs of applications converted to use PMEM because it mixes 4KB and 2MB page allocations internally.  Over time this leads to fragmentation, which causes the performance degradation we observed.  Thus, we subsequently switched to using ext4, which did not exhibit the performance degradation.
%%

%%
We used \texttt{perf}, a standard Linux performance utility, for measuring the processor performance.  From perf, we identified dramatically different data TLB miss rates, which in turn led us to finding both the default behavior of Linux and the sensitivity of page locality to DAX file system allocation policy.  \textbf{OS implementers make fundamental decisions we must understand to achieve good performance.}
%%

%%
We used \texttt{AEPWatch}, an open source Intel utility~\cite{IntelAEPWatch}, for measuring the performance behavior of the persistent memory controllers. Once we controlled for page size, this provided us with greater insight into the impact of read caching, write combining, and striping in the persistent memory controllers.  We found that \textbf{no single configuration produced consistently best performance.}
%%

%%
We hypothesized that software memory locality techniques would also yield improved performance.  To test this, we used the LLVM compiler tools, which include a polyhedral memory optimizer. The memory optimizer optimizes memory layout and code generation to improve data locality; Polybench was constructed to evaluate the effectiveness of polyhedral optimization.  In Table \ref{table:poly2} we show our results across DRAM, (non-interleaved) PMEM, and (interleaved) iPMEM, for both 4KB and 2MB page sizes.  Most PMEM configurations benefited from polyhedral optimization, which reflects the sensitivity of the system to locality. We found that non-interleaved PMEM benefitted most from polyhedral optimization, often exhibiting comparable performance to interleaved PMEM.  In only two cases did PMEM exhibit better performance than DRAM, likely due to the additional caching in the PMEM system itself. We omit results for eight tests where no memory configuration benefitted from polyhedral optimization.
%%

%%
Figure \ref{fig:polybench:dram-vs-ipmem-4k} was a fairer comparison of memory performance and a clear indication on the importance of locality.  We evaluated the impact of polyhedral memory optimization to corroborate our theory on the importance of data locality.  Figure \ref{fig:polybench:PMEM-vs-ipmem} was surprising to us because it suggested that locality was a more important factor than memory striping. Figure \ref{fig:polybench:all:time} demonstrated that in many cases polyhedral memory optimization improved data locality by as much as 80\%.  Table \ref{table:poly2} provides the specific timings for the 22 tests where polyhedral optimization improved performance for at least one memory type.  However, we were unable to identify any single factor that explained these exact results: we considered both data and instruction TLB miss rates, last level cache misses, data set sizes, and instruction counts. We suspect this is due to the generic nature of the polyhedral optimizer, which has no specific knowledge of the processor or memory characteristics on our test system.

\begin{comment}
%%
While prior work~\cite{yang2019empirical} combined these behaviors into a single measure (Effective Write Ration -- ``EWR'') our evaluation suggests that the performance profile is more nuanced than captured by this single ratio. Thus, we expect systems using PMEM will require workload specific evaluation across the range of configuration options to optimize performance.
%%
\end{comment}


